% !TEX encoding = UTF-8 Unicode

% This file contains an abstract of your thesis, with approximately 300-500 words
Understanding the mechanism of how cancer and metabolic disorders arise is important for finding new treatments for these diseases. Among different signaling pathways involved in these diseases, the insulin signaling holds a special role in modulating the cell growth and metabolic rate. Our lab's previous finding \cite{hahn_pp2a_2010} showed that the \textit{Drosophila} PP2A regulatory subunit B\textsuperscript{$\prime$} is a negative regulator of S6K, a major downstream component in the insulin and mTOR signaling. In order to understand better the function of this regulatory subunit and its implication in translational medicine, the mouse model and cell culture are employed to characterize the function of its mammalian homolog PPP2R5C in mouse, which is found to be the molecular checkpoint regulating the balance between glucose and lipid homeostasis in the mouse liver. Knockdown of PPP2R5C in the Hepa 1-6 cells and mouse primary hepatocytes shows that PPP2R5C could be a negative regulator for the triglyceride storage and glycolysis. Knocking down of PPP2R5C in several mouse cell lines results in the increased glucose uptake and glycolysis rate. Knockdown of PPP2R5C specifically in the mouse liver changes the mouse metabolism dramatically. The liver triglyceride and glycogen are increased while the liver cholesterol is decreased. Despite no change in the blood glucose level, the knockdown mice have better insulin sensitivity and glucose tolerance. During fasting and refed, the knockdown mice also have increased VLDL secretion from the liver. The microarray and \gls{qpcr} analysis on these samples also reveal multiple genes involved in the glycolysis and lipogenesis are up-regulated upon PPP2R5C knockdown, and most of these genes could be attributed to the increase in HIF1\textalpha{} and SREBP-1 activity. The substrate trapping for PPP2R5C identifies several master regulators in the metabolic process, such as AMPK, HIF1\textalpha{} and STAT3. \textit{in vitro} knockdown of PPP2R5C also shows increased AMPK activity and HIF1\textalpha{} phosphorylation. Interestingly, the mouse PPP2R5C is up-regulated in the \textit{db/db} mouse liver, which is a mouse model of type 2 diabetes. In addition, the human PPP2R5C is also elevated in the liver from type 2 diabetic patients. This study provides the new knowledge in PPP2R5C's metabolic function and interest in developing new drugs targeting the liver metabolism based on PPP2R5C.