% !TEX encoding = UTF-8 Unicode
\section{Molecular Biology}
\subsection{DNA digestion by restriction enzymes}
All plasmids and \gls{pcr} products were digested by restriction enzymes from Fermentas (Now part of Thermo Fisher Scientific). \gls{dna} was digested with restriction enzymes in 1$\times$FastDigest\textsuperscript{\textregistered} buffer for 2 hrs in 37\celsius{}. 

\subsection{Polymerase chain reaction (PCR)}
For Cloning \gls{pcr}, Taq polymerase from New England Biolabs was used and condition was as following.

Recipes:
\begin{center}
	\begin{tabular}{l | c}
	Components                 & 2.5$\times$Master Mix($\mu$L)\\
	\hline
	H\textsubscript{2}O        & 42\\
	Forward Primer (100$\mu$M)  & 0.2\\
	Reverse Primer (100$\mu$M)  & 0.2\\
	\gls{dntp} (10mM)                & 2\\
	10$\times$Standard Taq buffer    & 5\\
	Taq DNA polymerase         & 1\\
	\end{tabular}\\
\end{center}

For each reaction, bacterial clone was touched by pipette tip and washed in 10 $\mu$L sterile 1$\times$PBS and then 1 $\mu$L was add into 20 $\mu$L Master Mix. Then the \gls{pcr} was run with following condition.

PCR running condition:

\begin{center}
	\begin{tabular}{l c c c}
	Step       & Temperature & Time     & Cycles\\
	\hline
	Denaturing & 95\celsius & $3'$      & 1\\
	Denaturing & 95\celsius & $30''$    & \multirow{3}{*}{30}\\
	Annealing  & 60\celsius & $30''$    & \\
    Elongation & 72\celsius & $30''$    & \\
    Final	   & 72\celsius & $5'$      & 1\\
	
	\end{tabular}
\end{center}

For cloning Genomic \gls{dna} fragment from mouse liver \gls{gdna}, Expand\textsuperscript{\textregistered} Long Range DNA polymerase from Roche was used and condition was following:

Recipes:

\begin{center}
	\begin{tabular}{l | c}
	Components                         & 1$\times$Master Mix($\mu$L)\\
	\hline
	H\textsubscript{2}O                & fill up to 50 $\mu$L\\
	Template (gDNA,10--500ng/$\mu$L)     & 0.5\\
	Forward Primer (100$\mu$M)          & 0.3\\
	Reverse Primer (100$\mu$M)          & 0.3\\
	\gls{dntp} (10mM)                        & 2.5\\
	5$\times$Expand long rang buffer         & 10\\
	DMSO                               & 0,1,2 or 3\\
	Expand long range enzyme mix       & 0.7\\
	\end{tabular}\\
\end{center}

PCR running condition:

\begin{center}
	\begin{tabular}{l c c c}
	Step       & Temperature & Time     & Cycles\\
	\hline
	Denaturing & 92\celsius & $2'$      & 1\\
	Denaturing & 92\celsius & $10''$    & \multirow{3}{*}{10}\\
	Annealing  & 65\celsius & $15''$    & \\
    Elongation & 68\celsius & $60''$/kb & \\
    Denaturing & 92\celsius & $10''$    & \multirow{3}{*}{25}\\
	Annealing  & 65\celsius & $15''$    & \\
    Elongation & 68\celsius & $60''$/kb + $20''$/cycle    & \\
    Final	   & 68\celsius & $7'$      & 1\\
	
	\end{tabular}
\end{center}

For cloning \gls{cdna} from mouse, Hotstart\textsuperscript{\textregistered} polymerase from Qiagen was used and condition as following:

Recipes:

\begin{center}
	\begin{tabular}{l | c}
	Components                 & 1$\times$Master Mix($\mu$L)\\
	\hline
	H\textsubscript{2}O        & fill up to 50 $\mu$L\\
	Template (cDNA)             & 0.5\\
	Forward Primer (100$\mu$M)  & 1\\
	Reverse Primer (100$\mu$M)  & 1\\
	\gls{dntp} (10mM)                & 2\\
	10$\times$Hotstart\textsuperscript{\textregistered} Taq buffer    & 5\\
	Hotstart\textsuperscript{\textregistered} Taq DNA polymerase        & 0.5\\
	\end{tabular}\\
\end{center}

PCR running condition:

\begin{center}
	\begin{tabular}{l l c c}
	Step       & Temperature& Time         & Cycles\\
	\hline
	Denaturing & 95\celsius & $14'$        & 1\\
	Denaturing & 92\celsius & $30''$       & \multirow{3}{*}{35}\\
	Annealing  & 60\celsius & $30''$       & \\
    Elongation & 72\celsius & $60''$/kb    & \\
    Final	   & 72\celsius & $5'$         & 1\\
	
	\end{tabular}
\end{center}

\subsection{Agarose gel electrophoresis}
Agarose gel electrophoresis was employed to separate and purify \gls{dna} products from \gls{pcr}, plasmid digestion, etc.. Normally, 0.7--1.5\% ultrapure agarose gel was prepared in 1$\times$TAE by microwave mediated boiling. For separating ds/ss oligos, 4\% agarose gel was prepared with occasional shaking and mixing during boiling to help dissolving in TAE. DNA samples were mixed with 1/10 volume of 10$\times$GelRed dye stock solution. Running condition was 95--125V (constant voltage) depend on the size of the electrophoresis chamber used. Running time was 45 min as standard, and varied depending on the separation resolution needed (ie. separating size difference of 100--200 bp need more running time).

\subsection{Agarose gel purification}
The cut agarose gel was submitted to purification using NucleoSpin\textsuperscript{\textregistered} Gel and \gls{pcr} Clean-up kit from Macherey-Nagel. The purification procedure followed the manufacturer's instructions. 

\subsection{DNA ligation}
\gls{dna} ligation was done by mixing insert and linearized vector in 3:1 ratio (molarity ratio, 10:1 for blunt end) with following recipes.

Recipes:

\begin{center}
	\begin{tabular}{l | c}
	Components                        & 10 $\mu$L reaction\\
	\hline
	H\textsubscript{2}O               & fill up to 10 $\mu$L\\
	Insert                            & varies for insert size\\
	Vector                            & 1\\
	10$\times$T4 DNA ligase buffer(NEB)     & 1\\
	T4 DNA ligase(NEB)                & 0.5\\
	\end{tabular}\\
\end{center}

Ligation was usually done within 20 min at room temperature. For the ligation between dsDNA oligo and vector, ligation could also done at 16{\celsius} or 4{\celsius} overnight for better ligation efficiency.

\subsection{Transformation of E. \textit{coli}}
XL1 Blue competent cell was used for most \gls{gdna}, \gls{cdna} and oligo cloning. For cloning miRNA or \gls{shrna}, SURE 2 competent cell from Agilent Technologies and TOP10 from Life Technologies were used respectively. 1--4 $\mu$L ligation product was used for transformation and mixed with 50--100 $\mu$L competent cell suspension, then incubated on ice for 45 minute. Heat shock was done at 42{\celsius} for 30--45 seconds. Following incubation with 1 ml more \gls{lb} medium at 37{\celsius} was only needed for kanamycin resistant plasmid and occasionally for ampicillin resistance plasmid if transformation efficiency is very low. 

\subsection{Bacteria cultivation and plasmid purification}
Transformed bacteria with plasmid was incubated at 37{\celsius} and shake at 230 rpm overnight in 2--5 ml \gls{lb} medium or other special medium (NZY+ medium for SURE 2). Plasmid was isolated and purified with NucleoSpin\textsuperscript{\textregistered} Plasmid from Macherey-Nagel. All procedures were followed according to the manufacture's standard protocols. All purified plasmids were stored at -20{\celsius}. If sequencing is needed, All plasmids were sent to GATC Biotech for sequencing.

\subsection{Cloning and selection of shRNA or miRNA candidates}

\gls{shrna}s or \gls{mirna}s against common region in all PPP2R5C transcripts were designed on Invitrogen's BLOCK-iT\textsuperscript{\texttrademark} RNAi designer website (\href{http://rnaidesigner.lifetechnologies.com/rnaiexpress/}{BLOCK--iT\textsuperscript{\texttrademark} RNAi Designer}). 3 independent shRNA and 14 miRNA were selected based on their top positions on the rank list. Oligos for these shRNAs and miRNAs were synthesized from Sigma and cloned into Invitrogen's BLOCK-iT\textsuperscript{\texttrademark} adenovirus and adeno-associated virus system especially according to manufacturer's instruction. Knockdown efficiency was evaluated by co-expression of miRNA/\gls{shrna} with HA-tagged Variant 2 of PPP2R5C and checked by western blot.

\subsection{Cloning miR30-based shRNA for the inducible piggyBac shRNA system}

miR30-based \gls{shrna} was designed either using shRNA3 sequence designed at Invitrogen's RNAi website or new sequences predicted from Gregory Hannon's laboratory website for shRNA design (\href{http://cancan.cshl.edu/RNAi_central/RNAi.cgi?type=shRNA}{RNAi Central shRNA}). 3 independent shRNAs were synthesis and cloned into piggyBac transposase system with inducible shRNA expression (System Biosciences, PBQMSH812A-1). 

\section{Gene expression analysis}

\subsection{Tissue pulverization}
Frozen tissue was transferred into liquid nitrogen pre-cooled adapter sets with steal beads. The tissue was pulverized by TissueLyser II\texttrademark(Qiagen) for 1 min and at a frequency of 30 Hz (repeat if tissue was not homogeneous powder). Transfer pulverized powder into original tubes for these tissue samples.

\subsection{RNA isolation from tissue sample}
\textasciitilde50 mg of frozen tissue were weighted and transferred into a 2 ml RNase/DNase/Protease-free reaction tube containing 1 ml of Qiazol{\texttrademark} Lysis reagent and a stainless steel bead. The samples were homogenized using the TissueLyser II{\texttrademark} (Qiagen) for 1 min and at a frequency of 30 Hz. Lysate was transferred into a new 1.5 ml RNase/DNase free tube, and 200 $\mu$L chloroform was added into each tube. The mixture was further vortexed for 15 seconds at room temperature and then incubated under the hood for 15 minutes. To separate the RNA containing water phase, sample was centrifuged at 14, 000 rpm for 15 minutes at 4{\celsius}. Then 400 $\mu$L upper water phase was taken out and mixed with equal volume of isopropanol in a new tube, then another 14, 000 rpm 15 minute centrifugation at 4{\celsius} was applied to separate the RNA pellet. After washing with 70\% ethanol twice, RNA pellet was dried at room temperature till no visible solution, and then re-solubilized in 50 $\mu$L water. To increase the solubility, the RNA solution was incubated at 60{\celsius} for 10 min. The samples were stored at -80{\celsius} until further use.

\subsection{RNA isolation from cell sample}\label{sec:sec423}
1 ml Trizol{\texttrademark} Lysis reagent was directly applied onto cell in 6-well plate after removal of medium. Plate was then shaken on head-to-tail rotator for 2 min to allow complete lysis of the cell till no visible debris left. Then the lysate was transferred into 1.5 ml tube and 200 $\mu$L chloroform was added and mixed by vortexing for 15 seconds. RNA containing water phase was separated by centrifugation at 14, 000 rpm for 15 minutes at 4{\celsius}. Then 400 $\mu$L upper water phase was taken out and mixed with equal volume of isopropanol in a new tube, then another 14, 000 rpm 15 minute centrifugation at 4{\celsius} was applied to separate the RNA pellet. After washing with 70\% ethanol twice, RNA pellet was dried at room temperature till no visible solution, and then re-solubilized in 50 $\mu$L water. To increase the solubility, the RNA solution was incubated at 60{\celsius} for 10 min. The samples were stored at -80{\celsius} until further use.

\subsection{cDNA synthesis}
2--5 $\mu$L (depend on the concentration, make sure about 2 $\mu$g total RNA) RNA sample was used to synthesize \gls{cdna} from it. At first, RNA mix was prepared as following and heated at 65{\celsius} for 5 min:

\begin{center}
	\begin{tabular}{l | c}
	Components                   & 14.5 $\mu$L in total\\
	\hline
	H\textsubscript{2}O          & 10.5\\
	Oligo dT\textsubscript{20} (50 $\mu$M)  & 1\\
	\gls{dntp} mix(10mM)               & 1\\
	RNA sample                   & 2\\
	\end{tabular}\\
\end{center}

Then add 5.5 $\mu$L RT mix from following recipes, and mix well.

\begin{center}
	\begin{tabular}{l | c}
	Components                           & 5.5 $\mu$L in total\\
	\hline
	5$\times$RT buffer                         & 4\\
	Ribolock                             & 0.5\\
	Reverse Transcriptase(RevertAid)     & 1\\
	\end{tabular}\\
\end{center}

Then sample was put on \gls{pcr} machine from Bio-Rad (DNAEngine) at 50{\celsius} for 50 minutes and then inactivate enzyme in the reaction by heated sample up to 85{\celsius} for 5 minutes. After these steps, \gls{cdna} sample was stored at -20{\celsius}.

\subsection{Quantitative PCR (qPCR) analysis}
\gls{cdna} sample prepared as above was diluted in 1:60 in RNase/DNase/Protease-free water. Then 4 $\mu$L of diluted cDNA as template for \gls{qpcr} analysis. Working master mix for qPCR was made from 5 $\mu$L 2$\times$Maxima SYBR Green/ROX qPCR Master Mix (Fermentas), and 1 $\mu$L oligo mix (2.5 $\mu$M each primer). The \gls{pcr} reaction mix was transferred to a MicroAmp{\texttrademark} Optical 96-well reaction plate (Applied Biosystems). All reactions were performed in technical duplicates. Quantitative PCR was performed using a StepOne Real Time \gls{pcr} System (Applied Biosystems, now part of Thermo Fisher Scientific). Gene expression levels were calculated by $\Delta \Delta C_t$ method.

\subsection{Microarray analysis of mouse tissue and cell sample}
Expression profile analysis by microarray was done for RNA sample from Hepatoma cell line Hepa 1-6, primary mouse hepatocytes and mouse liver tissues with infection by adenovirus packaged with control \gls{shrna} or shRNA targeting common region of all mouse PPP2R5C splicing isoforms at MOI (\textbf{M}ultiplicity \textbf{o}f \textbf{I}nfection) of 100. RNA was isolated as protocol in Section~\ref{sec:sec423} and send to DKFZ's in-house Genomics \& Proteomics Core Facility for microarray analysis. RNA sample was analyzed on Bioanalyzer (Agilent Technologies) for quality control and then submitted for \gls{cdna} synthesis and microarray analysis using MouseWG-6 v2.0 Expression BeadChip Kit from Illumina Inc. The raw data from the core facility was sent back and further processed and analyzed on DKFZ's in-house Chipster server  \cite{kallio_chipster:_2011}. Alternatively, raw data was processed in R and analyzed with limma package in R. 

\section{Cell biology}

\subsection{Cell culture for Hepa 1-6, HEK293T, HEK293A, Hela Cells}
All cell lines were maintained and propagated in Dulbecco's Modified Eagle Medium with 4.5 g/L glucose (\gls{dmem}), 10\% fetal calf serum (\gls{fcs}) and 1$\times$penicillin/streptomycin (100 IU and 100 $\mu$g/mL). HEK293A and HEK293T cells also required 1$\times$Non-Essential Amino Acids (\gls{neaa}). Cell was split in 1:10 twice per week. Experiments involving eukaryotic cells were performed under sterile conditions. Media and reagents were preheated to 37{\celsius} prior to use. All cells were cultivated at 37{\celsius}, 5\% CO\textsubscript{2} and 95\% humidity in 96-well, 24-well, 12-well, 6-well, 10 cm or 15 cm cell culture dishes.

\subsection{Transfection assay}
For the transfection plasmid into cells, Effectene\textsuperscript{\textregistered} Transfection Reagent from Qiagen was used according to standard protocols provided in kit's instruction. Medium with transfection reagent was exchanged with fresh medium after overnight incubation. Additional 1-2 day was needed for proper expression of exogenous genes.

\subsection{Mouse primary hepatocyte cultivation}

For cultivated mouse primary hepatocytes, two different sources of them were kindly provided by Prof. Herzig's and Prof. Klingm\"uller's lab especially. These two sources of hepatocytes were prepared in each lab with the same protocol of isolation. However, hepatocytes from Prof. Klingm\"uller's lab were also counted for living cell by trypan blue staining and exactly 1 million cell were seeded on 6-well plate. Here only the protocol from Prof. Herzig's lab is described for simplicity.

Mouse primary hepatocytes were isolated and \textit{in vitro} cultivated as standard procedure in Prof. Herzig's lab  \cite{klingmuller_primary_2006}. Male 8--12 week old C57Bl/6J mice were housed for 1 week and then anesthetized by intra-peritoneal injection of 5 mg 10\% ketamine hydrochloride/100 mg body weight and 1 mg 2\% xylazine hydrochloride\slash100 mg body weight. When there was no response from pressing mouse foot, it was then allowed to open the abdominal cavity. The liver was then perfused with HANKS I buffer via the portal vein for 5 min at 37{\celsius} and subsequently with HANKS II buffer for 5--7 min until complete disruption of the liver structure is visible (color change from red to pale). Then the liver was cut out and the liver capsule was removed and washed gently in adhesion buffer (recipe in Table~\ref{tab:tab8}) until no visible cell was left attached onto the capsule. Then the liver cell suspension was filtered through a 100 $\mu$m mesh fitted into 50 mL Falcon tube (BD Biosciences). Hepatocytes were washed twice and gently collected by centrifugation at 37.5$\times$g at room temperature. Cell suspension from one mice was equally distributed in collagen I-coated 6-well plates (roughly 1 million cell for complete coverage) without checking the cell viability by trypan blue staining. Hepatocytes were infected with recombinant adenoviruses (MOI = 10, 100 or 200) 4 hours after seeding and harvested for gene expression analysis or submitted for Triglyceride, free fatty acid, glucose, and lactate measurement after 48 or 72 hours later.

\subsection{PP2A substrate trapping in Hepa 1-6}
Protein-protein interaction mapping by biotinylation \cite{roux_promiscuous_2012} was adapted to discover new substrate of PP2A holoenzyme with PPP2R5C as regulatory B$'$ subunit. In order to stabilizing interaction between substrates and PP2A, a phosphatase dead mutant of C catalytic subunit of PP2A was also co-expressed together with promiscuous biotin ligase tagged PPP2R5C Variant 1. The mechanics behind this method design is call substrate trapping. It has been successfully used to find several protein phosphatases' substates \cite{boubekeur_new_2011, flint_development_1997,wu_identification_2006}. 2 $\mu$g of each plasmid was transfected into Hepa 1-6 cell in 6-well plate with 50 $\mu$M biotin in medium. After 24 hrs of expression, cell was washed twice in PBS and lysed in \gls{bioid} lysis buffer 1. Then equal volume of BioID lysis buffer 2 was added and mixed. Clarified supernatant was collected and incubated with 100 $\mu$L Dyna-Beads (MyOne Streptavidin C1 from Life Technologies) overnight. One the second day, beads were collected and washed twice with BioID wash buffer 1 on a magnetic separator (DynaMag{\texttrademark}--Spin Magnet). The washing was repeated once with BioID wash buffer 2, once with BioID wash buffer 3 and twice with BioID wash buffer 4 (all buffers used in substrate trapping are listed in Table~\ref{tab:tab8}). Finally, protein was eluted from beads by BioID elution buffer. Protein sample was either submitted for western blot cross validation or mass spectrometry identification, which is performed at DKFZ's in-house proteomics core facility. 

\subsection{Luciferase assay}

All promoter reporters used in this project were cloned into pGL3 promoter from Promega. Transfection of luciferase reporters into hepa 1-6 cell was titrated and optimized for 96-well or 24-well plate. Cell was lysed in either 50 $\mu$L or 200 $\mu$L passive lysis buffer from Promega's Duo-luciferase reporter assay system, and renilla luciferase was used as control. 

\subsection{Inducible shRNA stable cell line generation}

Hepa 1-6 cell was transfected with piggBac transposase expression plasmid and \gls{shrna} containing plasmid using Effectene\textsuperscript{\textregistered} transfection reagents from Qiagen. After 6 hour post-transfection, cell was selected under 3 $\mu$g \slash mL puromycin until clones were formed under microscope check. shRNA integrated cell was either submitted for continuous selection for two weeks or picked as single clone for continuous selection for additional two weeks. Generated stable cell line was submitted for 1 week puromycin selection every month during culture.

\subsection{FACS analysis of 2NBDG uptake}

Stable cell line with inducible \gls{shrna} or empty hepa 1-6 cell was cultivated and induced for 3--4 days and then starved in serum-free \gls{dmem} overnight. Then these cells were sensitized in KRPH buffer (20 mM HEPES, 5 mM KH\textsubscript{2}PO\textsubscript{4}, 1 mM MgSO\textsubscript{4}, 1 mM CaCl\textsubscript{2}, 136 mM NaCl, 4.7 mM KCl, adjust pH to 7.4 (from pH 5.1 to 7.4)) for 1 hour and the mixed with 2NBDG up to 100 $\mu$M, 20 min to allow glucose analog uptake. Uptake was stopped by washing with PBS for 3 times and then digested with 0.25\% trypsin for 3 min. Digestion was stopped by adding equal volume of FBS (fetal bovine serum from PAA). All the cells were suspended and washed in PBS with 2\% FBS for 3 times before FACS measurement. 2NBDG intensity was recorded in the same channel for GFP on BD's FACSCanto\textsuperscript{\texttrademark} II. FACS data was analyzed either in FlowJo or R.


\section{Virus production for mouse \textit{in vivo} knock-down}

\subsection{shRNA packaging Adenovirus construction and production}

\subsubsection{Adenovirus with shRNA-NC{\slash}3 construction}
The BLOCKiT\textsuperscript{\texttrademark} Adenoviral RNAi System from Life Technologies was employed to clone and package control \gls{shrna} or shRNA3 (targeting common region of all splicing isoforms of \textit{PPP2R5C}) from ds oligos. Oligonucleotide sequences were designed using Invitrogen’s online RNAi Design server \cite{_invitrogen_2014}. Two complementary ssDNA oligos against the target gene sequence were ordered from Sigma, and re-suspended as 200 $\mu$M in water. Then oligo mixes with 1$\times$annealing buffer was denatured at 98{\celsius} and annealed from 98{\celsius} to 90{\celsius}, then hold for 5 minutes, and annealed again from 90{\celsius} to room temperature. The annealed ds oligo products were checked by 4\% agarose gel and cloned into the pENTR\textsuperscript{\texttrademark}{\slash}U6 vector according to the manufacturer’s instructions. The sequence verified constructs were recombined with the pAd{\slash}BLOCK-iT\textsuperscript{\texttrademark} DEST vector, which contains the adenovirus serotype 5 \gls{dna} but not the E1 and E3 genes that are required for viral replication. The viral vector containing the shRNA sequence was linearized by restriction digest using the enzyme PacI and transfected into HEK239A cells using Lipofectamine 2000 reagent according to the manufacturer’s instructions. HEK293A cells express the viral E1 and E3 genes necessary for viral lysis, which allows the virus to be propagated in culture medium. Viral plaques become visible from 6 to 10 days after transfection and cell monolayer started to form plaques. When \textasciitilde70\% of cells were round and detaching, it was time to harvest them.

\subsubsection{Adenovirus harvesting}\label{sec:sec4412}
HEK293A cells containing adenovirus were harvested from the medium after complete detachment of all round infected cells. The medium was collected from up to 20$\times$15 cm culture dishes and centrifuged at 2, 000 rpm, 4\celsius{} for 10 min. The supernatant was then discarded and the pellet was resuspended in 4 mL PBS-TOSH buffer inside 15 mL Falcon tube. The cell pellets in Falcon tube were frozen in liquid nitrogen and subsequently thawed at room temperature on vortex for 3 times for maximal cell lysis and adenovirus releasing. After 3 times lysis the cell suspension was centrifuged at 2, 000 rpm, 4{\celsius} for 10 minutes. The clarified supernatant was then stored at -80{\celsius} or directly submitted for a CsCl gradient purification.

\subsubsection{Adenovirus purification by CsCl gradient}
Virus lysates from Section~\ref{sec:sec4412} was filled with PBS-TOSH upto 20 mL final volume. CsCl gradients were prepared in ultracentrifuge tubes (Beckmann Polyallomer 25mm$\times$89 mm) and were weight-balanced after addition of each solution. At first, \textasciitilde9 ml 4 M CsCl was added, then \textasciitilde9 ml of 2.2 M CsCl was added and finally the viral lysate was carefully added on top in one liquid droplet by one fashion. In the end, there should be 3 different visible gradient layers. These gradients were centrifuged at 24, 000 rpm, 4{\celsius} in ultracentrifuge XL-70 (Beckmann) with a SW28 swing bucket rotor for 2 hrs. After ultracentrifugation a visible white band representing the concentrated adenovirus fraction was formed between the 4 M and 2.2 M CsCl gradients. The band was collected by inserting into the tube with 25G needle connected with 5 ml syringe. The collected virus (\textasciitilde{}3 ml) fraction was then diluted with equal volume of saturated CsCl and changed into a 12 mL ultracentrifuge tube (Beckmann Polyallomer 14mm$\times$89 mm). \textasciitilde2 ml of 4 M CsCl and 2.2 M CsCl were utilized again to form gradients. Then a centrifugation at 35, 000 rpm, 4{\celsius} was applied in an SW41 Ti swing bucket rotor for 3 hrs. The viral fraction was visible again as a white band between the 4 M and 2.2 M CsCl gradients. The fraction (\textasciitilde{}700 $\mu$L) was collected using a needle and 1 ml syringe. Finally, the viral fractions were dialyzed (Spectra{\slash}Por\textsuperscript{\textregistered} Biotech, MWCO 15,000, 10 mm diameter) in 1 L 1$\times$PBS with 10\% glycerol (v{\slash}v) for 2 times (1 and 24 hrs each) at 4{\celsius}. After dialysis, 50--200 $\mu$L adenovirus in PBS were aliquoted and stored at -80{\celsius} before use.

\subsubsection{Adenovirus titration}
Adenovirus titer was determined by the Tissue Culture Infectious Dose 50 (TCID50) assay. For titer measurement, $10^{4}$ HEK293A cells/well were cultivated in 100 $\mu$L \gls{dmem} medium with 2\% \gls{fcs} (v{\slash}v), 1$times$\gls{p/s} and 1\% \gls{neaa} in 96-well plate. In order to have more accurate TCID50 calculation, technical duplicates of 96-well plate were required for titer measurement. Cells were completely attached to the plate after 4 hours seeding. During cell attachment, serial dilutions of the adenoviruses (1.65 ml for each dilution, from $10^{-6} - 10^{-13}$) were prepared in the same medium as that for cultivation. 100 $\mu$L of each virus dilution was added to ten wells and 100 $\mu$L of medium without virus was added to the rest two well in the same row in 96-well plate as negative control wells. The infected cells were cultivated for 10 days for continuous monitoring plaque formation everyday. At day 10, the number of wells with at least one plaque was investigated under microscope for each dilution, and the the titer was calculated with following formula:

\begin{align*}
Ta & = \text{viruses per 100 $\mu$L}  = 10^{1 + (S - 0.5)}\\
S &  = \text{\parbox[t]{8 cm}{the sum of all positive wells starting from the $10^{-1}$ dilution, whereby 10 positive wells correspond to the value 1.}}  \\
T &  = \text{viruses per 1 ml}  = 10 \times Ta
\end{align*}

\subsection{miRNA packaging Adeno-Associated Virus construction and production}

\subsubsection{miRNA containing AAV construction}
\textit{PPP2R5C}-specific or non-targeting scramble control \gls{mirna} was cloned into Invitrogen's adeno-associated virus system for long-term knockdown of \textit{PPP2R5C} \textit{in vivo}. The Oligonucleotide sequences were designed using Invitrogen’s online RNAi Design server \cite{_invitrogen_2014}, and listed in Table~\ref{tab:tab8}. The oligos were then synthesized from Sigma, then annealed and cloned into the pcDNA6.2-GW{\slash}EmGFP-miR vector. Later these oligos were sub-cloned into pdsAAV-LP1-EGFPmut \gls{AAV} vector \cite{kulozik_hepatic_2011} between the BglII and SalI sites. The pdsAAV plasmids with miRNAs were co-transfected into HEK293T cells for AAV production, together with the pDG$\Delta$VP helper plasmid \cite{grimm_novel_1998} and a mutated p5E18-VD2/8 expression vector \cite{gao_novel_2002} encoding AAV2 rep and a mutated AAV8 cap protein. For virus production, cells from 6$\times$15 cm culture dish with 90\% confluence were scrapped and resuspended in 1100 ml DMEM medium (with 10\% FCS, 1$times$\gls{p/s}). 1000 ml of the cell suspension was transferred to a 10$\times$cell-stack chamber for first round virus production. And the left 100 ml was transferred to a 1$\times$cell stack chamber as cell source for second round virus production in 10$\times$cell-stack (could also used as control chamber for checking cell density). 2 days after plating, the cells in 10$\times$cell-stack chamber were reaching 90\% confluent and co-transfected with the plasmids encoding the viral genes using the \gls{pei} method in the amounts described below. 
\begin{center}
	\begin{tabular}{c c}
	Plasmid                   & amount in $\mu$g\\
	\hline
	AAV-miRNA expression vector          & 395\\
	p5E18 VD2/8 helper plasmid  & 497\\
	pDG$\Delta$VP helper plasmid               & 1353\\
	\end{tabular}\\
\end{center}

After 1 or 2 day transfection, cells were ready for harvesting and washed with 1$\times$PBS once. Then 10 or 100 mL trypsin was added to 1$\times$ or 10$\times$cell-stack for 5 minute digestion at 37{\celsius} respectively. 40 ml or 350 mL full DMEM medium with serum was added into for quenching trypsin digestion. And then the cells were transferred into a 50 mL falcon tube, or a 500 mL conical tube. For cells in 50 mL falcon tube, they were repopulated in used 10$\times$cell-stack chamber for the second round cell transfection and harvesting. The cells in 500 mL conical tube were spun down at 2000 rpm for 10 min. The supernatant was discarded and the cell pellets were resuspended in 8 mL lysis buffer (150 mM NaCl and 50 mM Tris-HCL, pH 8.5) and transferred in 15 Falcon tube, frozen in liquid nitrogen and stored at -80{\celsius}.

\subsubsection{AAV crude lysate preparation}

\gls{AAV} lysates from step above were thawed at 37{\celsius} until half frozen and half suspension mixture formed. Then the votexing was employed to have complete thawing. The virus containing supernatant was collected by 10 minute centrifugation at 3500$\times$g. The cell pellets were remixed in 4 mL lysis buffer and then snap-frozen in liquid nitrogen. The freezing-and-thaw cycle was repeated for 3 times to have maximal cell lysis. The final thawing step was employing 1 minute sonication before centrifugation for better cell lysis. Finally, suspensions from all steps were pooled and digested with benzonase (50 U/ml) for 30 min at 37{\celsius} to remove any transfected plasmid and naked DNA. This virus lysate was then centrifuged at 4{\celsius} and 3,500 g for 10 min to remove the pellet and then stored at -80{\celsius} until further use.

\subsubsection{AAV iodixanol gradient purification}
\gls{AAV} crude lysates were further purified by two-step iodixanol gradient. 4 gradients with different concentration were prepared as in table as follow (quantity for 15 gradients):

\begin{center}
	\begin{tabular}{l r r r r}
	Iodixanol Gradient  & \%15 & \%25 & \%40 & \%60\\
	\hline
	OptiPrep    & 17.5 mL & 31.2 mL & 40 mL & 60 mL\\
	PBS-MK-NaCl & 52.5 mL &         &       &      \\
	PBS-MK      &         & 43.8 mL & 20 mL &      \\
	0.5\% Phenol Red &    & 187.5$\mu$L  &  & 150$\mu$L\\
	Total       & 70 mL   & 75 mL   & 60 mL & 60 mL\\
	\end{tabular}\\
\end{center}

Lysates were transferred into centrifugation tube via Pasteur pipette. 4 different gradients were sequentially under-layered through Pasteur pipette. Then centrifugation tube was sealed and \textasciitilde1 mL air bubble was left inside. Gradients were centrifuged at 50, 000 rpm, 2.5 hours and 10{\celsius} in 50.2Ti rotor. Purified virus fraction was taken out by inserting 20G needle into 60\% layer and collecting roughly 3.5 mL 40\% gradient fraction. In the second step gradient purification, only 25\%, 40\% and 60\% gradient were used for purification. Other procedures were the same as in step 1. After two round gradient purification, virus fraction was dialyzed against PBS overnight with 3 changes of PBS. Then virus fraction was transferred into Vivaspin-6 tube and spun and resuspended at 2000--4000 rpm for 3--5 min until the final volume is 1000-1500 $\mu$L. Finally, virus elute was aliquoted and stored at -80{\celsius} until further use.

\subsubsection{AAV titration}

5 $\mu$L virus solution was mixed with 5 $\mu$L H\textsubscript{2}O and 10 $\mu$L NaOH and then incubated at 55{\celsius} to allow complete release of the viral genome. Then this solution was neutralized by adding 10 $\mu$L HCl and diluted by adding 970 $\mu$L  H\textsubscript{2}O. Finally, 5 $\mu$L was submitted to \gls{qpcr} analysis with probe for GFP sequence in \gls{AAV} genome. For quantification, the standards were prepared by diluting \gls{AAV} genome plasmid from 10\textsuperscript{13} copies/mL step-wise into 10\textsuperscript2.

\section{Metabolite measurement in celluar and tissue samples}

\subsection{Glucose consumption in Hepa 1-6 and primary hepatocytes}

Hepa 1-6 cells or mouse primary hepatocytes were cultured as condition described before. At day 0, adenovirus packaged with scramble \gls{shrna} or shRNA3 (targets all PPP2R5C transcripts) were incubated with cell for 3-day infection experiment at MOI of 10, 100, or 200. After 24 hours, media with virus was washed away by 3 time washing with fresh medium. Cell media was replaced everyday, and media for last 24 hour infection was collected for glucose consumption assay. Glucose concentrations from 2.5 $\mu$L of these media or control media (fresh media) were measured by Glucose HK assay kit from Sigma, and relative glucose consumption was calculated by subtracting glucose concentration from fresh medium. 

\subsection{Lactate production} 

The same media collected for glucose consumption assay was also used for lactate production assay. 2 $\mu$L of each medium was submitted for lactate concentration measurement by using a lactate kit from Roche (D-Lactic acid/L-Lactic acid, Cat. No. 11 112 821 035). And lactate production rate in 24 hour was calculated from relative lactate production compared with fresh medium.

\subsection{Free fatty acid measurement}

Sample for free fatty acid measurement was prepared differently among various samples. For media sample, all the lipid content was enriched by Methanol-Chloroform method  \cite{folch_simple_1957}. 2 volume of methanol:chloroform mix (2:1 in volume ratio) was mixed well with 1 volume of media (if media does not contain Triton X-100, add 10 $\mu$L chloroform:Triton X-100 mix (1:1 in volume ratio)), and then was shaken at room temperature at 250 rpm to allow complete extraction of lipid fraction. Then these solutions were spun down at maximal speed on a desktop centrifuge for 5 min, room temperature to separate the water:methanol phase and the chloroform phase. Then lipid fraction was collected from the lower chloroform phase, and mixed with 0.4 volume of 0.9\% NaCl for cleaning. Final spin on this solution will separate the chloroform phase with the water phase on the top. Final lipid fraction was collected and dried in a speed-vac for at least 3 hours until there was a Triton X-100 pellet formed. Finally, pellet was dissolved in water and submitted to fatty acid measurement using the kit from Cayman Chemical (Free Fatty Acid Fluorometric Assay Kit) under the instruction manual from the manufacturer. 

For cellular samples lysed by IP lysis buffer (150 mM NaCl, Tris pH 7.5, 1\% Triton X-100), they could be submitted to fatty acid measurement directly if the fatty acid concentration was within the range of standard curve. If not, then these samples must be enriched as before. For tissue sample such as mouse liver, The sample is needed to be pulverized and weighted before suspending in methanol:chloroform mix (2:1 in volume ratio). For 100 mg liver tissue powder, 1.5 mL mix was added and shaken at room temperature for 20 min. The following procedure for collecting lipid fraction was similar with these for media samples.

For serum samples, 2 $\mu$L of serum was subjected to fatty acid measurement by \gls{nefa} HR kit from Wako Chemicals. 2 $\mu$L serum sample or different amount of standard (1 mM/L oleic acid, 0.5 - 5 $\mu$L) was first mixed with 200 $\mu$L R1 reagent, and incubated at 37 \celsius{} in Tecan microplate reader Infinite M200 for 2 min to read the absorbance 546 nm as blank. Then 100 $\mu$L R2 reagent was added at 3 min, and absorbance at 7.5 min was recorded as final reading. Serum \gls{nefa} was calculated from back calculation from the standard curve.

\subsection{Triglyceride measurement}

Sample preparation procedure for triglyceride measurement were the same as the ones for free fatty acid measurement. After sample was lysed or enriched, normally 2--20 $\mu$L sample was mixed and incubated with 5 $\mu$L lipase solution (10 mg/mL) at 37 \celsius{} overnight to release the glycerol from triglyceride. Then free glycerol content was measured by free glycerol assay kit from Sigma. Triglyceride concentration was calculated and normalized to the total protein content in the very same sample.


\subsection{Cholesterol measurement}

Cholesterol was measured from the same sample for triglyceride or free fatty acid measurement. For serum sample, 2--4 $\mu$L serum was directly submitted for measurement. All cholesterol measurements were done using cholesterol (liquid) assay kit from Randox Laboratories (Cat. No. CH201) according to the standard protocol shipped with the kit. Cholesterol concentration was calculated from control with known concentration.
	

\subsection{Glycogen content determination}

For measuring glycogen from cellular lysate, such as Hepa 1-6 in 6 well plate, cell was first lysed in 500 $\mu$L IP buffer and 30 $\mu$L of the lysate was mixed with 1 $\mu$L amyloglucosidase solution (14 U/$\mu$L) overnight at 37 \celsius{}. For tissue sample such as mouse liver, \textasciitilde50 mg liver was weighted (recording the weight for normalization) and homogenized in 1 mL 30\% KOH with Qiagen tissue lyser. Then the homogenate was incubated at 95 \celsius{} for 30 min and clarified by maximal centrifugation in a desktop centrifuge for 10 min. Supernatant was collected and mixed with 1.5 mL 95\% ethanol to precipitate glycogen. Glycogen pellet was collected by spinning at 3000$\times$g for 20 min and then washed with 95\% ethanol, dried at room temperature. Then the pellet was dissolved in 500 $\mu$L water (in case there was solubility issue, heat to 37 \celsius{} for 30 min). 5 $\mu$L glycogen solution was digested using 295 $\mu$L amyloglucosidase solution (30 U/mL in 0.2 M NaAc pH 4.8) and then neutralized with 6 $\mu$L 30\% KOH. 10 $\mu$L of the digested glycogen was submitted to glucose measurement using the same kit for glucose consumption assay.


\subsection{Ketone body measurement}

Total ketone body and 3-hydroxybutyrate were measured using Autokit total ketone bodies and 3-HB kit from 
Wako Diagnostics. 2 $\mu$L of each serum was taken for measurement. And procedures were the same according to the manufacturer's instruction. All the reading was done in 96 transparent well and recorded on Tecan infinite M200 at 405 nm.


\subsection{ATP measurement}

Intracellular ATP level was measured using ATPlite\textsuperscript{\texttrademark} from Perkin Elmer. 50 $\mu$L of Hepa 1-6 lysate (lysis in 500 $\mu$L IP buffer for 10\textsuperscript{7} cell) was mixed with equal volume of reagent and read for luminescence intensity within 20 min on Tecan infinite M200.

\subsection{Intraperitoneal glucose tolerance test (IPGTT)}

Mice were starved for 6 hours before injecting 2g glucose/kg body weight intraperitoneally. Blood glucose was collected from cutting at tail and measured right after injection. Then 50 $\mu$L blood was also collected from the same cutting point for measuring insulin in the serum. At time point 20, 60, 90, 120, 150 min, blood glucose concentration was measured and extra 50 $\mu$L blood was collected for later insulin measurement at time point of 20 and 60 minute. Blood was incubated at 4 \celsius{} to allow separation of serum and serum was collected by spinning at 5000 rpm for 30 min. All blood glucose measurement was done by using a glucose measurement kit from One Touch Glucose Monitor (Lifescan).

\subsection{Serum insulin measurement}

Mouse insulin in the serum was performed using mouse insulin \gls{elisa} kit from Alpco. 5 $\mu$L serum was used for mouse serum from random fed mice. For \gls{gtt} assay or serum from fasting mouse and refed mice, 25 $\mu$L was taken for measurement. Insulin concentration was calculated from 5 parameter logistic regression of standard curve data (0.188--6.9 ng/mL) in R package drc. All the reading was done in 96 transparent well and recorded on Tecan infinite M200 at 450 nm.

\subsection{Serum lipoprotein particle analysis}

200 $\mu$L of pooled serum from 5 or 6 mice (40 or 33.3 $\mu$L for each mouse) and 100 $\mu$L 1$\times$ PBS were mixed and spun for 2 min at 10, 000$\times$g, 4 \celsius{} to clarify. And then the 300 $\mu$L mixed solution was subjected to \gls{fplc} separation on Superose\textsuperscript{\texttrademark} 6 10/300 GL column (GE Healthcare) in 25 mL PBS at flow rate of 0.5 mL/min. Lipoprotein particle peaks were monitored with UV 280nm. \gls{vldl}, \gls{ldl} and \gls{hdl} eluted at 7-8 mL, 12-14 mL and 16-18 mL respectively. 0.5 mL/Fraction was used to collect each fraction, and then 160 and 40 $\mu$L from each fraction was used to determine the triglyceride and cholesterol content per fraction respectively.

\subsection{Serum ALT measurement}

Serum ALT (Alanine Aminotransferase) activity was measured by the Infinity ALT/GPT Reagent (Thermo Scientific). 5 $\mu$L serum or H\textsubscript{2}O was added into 100 $\mu$L ALT reagent on 96-well transparent plate (SARSTEDT). Then the microplate was incubated at 37 \celsius{} and recorded UV 355 nm on the microplate reader (Infinite\textsuperscript{\textregistered} M200, Tecan) at time of 1 min and 9 min. Then the ALT activity was calculated as following equation.

\begin{align*}
ALT & = \frac {\Delta A_{1 min - 9 min}}{8} \times 11029.4 \text{(U/L)}
\end{align*}

\subsection{Seahorse analysis of glycolysis in Hepa 1-6}

Pre-split hepa 1-6 cell was washed once with 180 $\mu$L assay medium (\gls{dmem} (Sigma D5030) with 143 mM NaCl, 2 mM L-Glutamine, pH 7.35$\pm$0.05. Adjust pH at day of assay.) and incubated with 175 $\mu$L assay medium at 37 \celsius{} for 1 hour in 96-well plate from XF96 glycolysis stress kit (Seahorse Bioscience). Then 25 $\mu$L of each glucose (10 mM in assay medium), oligomycin (2.5 or 1 $\mu$M in assay medium) and 2-deoxyglucose (100 mM in assay medium) was injected into plate reservoir and then glycolysis rate was recorded on XF96\textsuperscript{e} Extracellular Flux Analyzer from Seahorse Bioscience. All the data collection and analysis was done in built-in software for XF96. Data normalization was done by counting average nuclei number in Cell Profiler from DAPI staining and normalization to average nuclei count. 

\section{Biochemical methods}

\subsection{Protein expression in bacteria}

Variant 3 of PPP2R5C was sub-cloned into 6 $\times$ His tag purification system. Expression plasmid was transformed into 4 different bacteria strains by electroporation in Bio-Rad's Gene Pulser Xcell\textsuperscript{\texttrademark} Electroporation Systems, include Lucigen, Rosetta, RP and RIL. Strain with the highest induction under IPTG was selected for protein production. 1 L bacteria culture was shaken at 37 \celsius{} until O.D. 600 exceed 2 and then IPTG induction was done at 18 \celsius{} overnight at 1 mM. Protein purification on 6 $\times$ His tag resin from Qiagen were performed according to suppliers' instruction.

\subsection{Antibody production and purification}

Purified PPP2R5C was dialyzed in PBS with 5\% glycerol overnight and then concentrated by Vivaspin 2 with 3000 MWCO until protein concentration reached 1 mg/mL. Antigen was mixed with equal volume of Freund's adjuvant complete and injected 250 $\mu$L per guinea pig at each 3--4 weeks. Around 50 $\mu$L blood was collected at 1 week after injection. Serum was separated and tested for specificity for antigen at 1:200--1000 dilution in western blot.

\subsection{SDS-PAGE and western blot}

All protein samples, including lysates from cell, tissue or serum fractions, were denatured in 1$\times$Laemmli buffer (60 mM Tris-Cl pH 6.8, 2\% SDS, 10\% glycerol, 5\% $\beta$-mercaptoethanol, 0.01\% bromophenol blue) and boiled at 95 \celsius{} for 5 min. SDS-PAGE gels were prepared at different percentage using Tris-Glycine gel system. Gel was running at 20 mA/gel with 200 V limit for 1 hour. And then gel was washed in transfer buffer (25 mM Tris, 192 mM Glycine, 20\% Methanol) shortly before assembled into wet transfer sandwich with Whatman paper and 0.22 $\mu$M nitrocellulose membrane from GE Healthcare. Protein was transferred onto nitrocellulose membrane at 100 V for 1 hour at 4 \celsius. Afterwards, protein was visualized by Ponceau S staining (0.2\% in 3\% TCA) and blocked with 5\% \gls{bsa} or skim milk (Sigma) for 1 hour. Then the membrane was washed with PBST (1$\times$PBS, 0.1\%Tween-20) for 3 times, 10 min each. Incubation with primary antibody was done at 4 \celsius{} overnight. Then the membrane was washed for 3 times, 10 min each. Secondary antibody was diluted in 5\% skim milk and incubated with membrane at room temperature for 1 hour. Finally, the membrane was washed 3 times, 10 min each, and developed with ECL reagents from Thermo Scientific.

\subsection{Phos-tag\texorpdfstring{\textsuperscript{\textregistered}}{\textregistered}  analysis of phosphorylated proteins}

For protein separation using Phos-tag\textsuperscript{\textregistered} analysis, all protein lysate samples were cleaned by Methanol-Chloroform precipitation  \cite{wessel_method_1984} and re-solubilized in 1$\times$Lamelli buffer. 25 $\mu$M Phos-tag\textsuperscript{\textregistered} was used to incorporate with 8\% SDS-PAGE gel. Other procedures were performed under the instruction manual of Phos-tag\textsuperscript{\textregistered}. 

\subsection{Immunoprecipitation}

Total protein lysates were prepared by lysing cell in IP buffer (50 mM Tris, pH 7.5, 150 mM NaCl, 1\% Triton X-100) for 15 minutes on ice and then collecting supernatant after centrifugation of 14, 000 rpm at 4 \celsius{} for 15 minutes. Then lysates were incubated with 1 $\mu$L antibody (0.5 mg/ml) for 3 hours at 4 \celsius, and 30 $\mu$L Protein A Agarose slurry (Roche) for additional 0.5 hour. Agarose beads were washed in cold IP buffer for 3 times with centrifugation at 2000 rpm, 4 \celsius{} for 1 minute to discard the supernatant. Final IP fraction was eluted in 2$\times$Lamelli buffer and boiled at 95 \celsius{} for 5 minutes.

\section{Animal experiments}

8--10 week C57BL/6J male mice was purchased from Charles River Laboratories and maintained with unlimited water and normal chow food at 12 hour light--dark cycle. After 1 week adaptation, mice were tail-injected 100 $\mu$L control or knockdown \gls{AAV} diluted in 1$\times$PBS. After 7 week infection, mice were subjected to \textit{ad libitum} feeding, 16 hour fasting or 16 hours fasting + 6 hour refeeding.

\section{Data analysis and plotting}

Data for mouse serum and liver metabolites were imported and analyzed in R  \cite{r_core_team_r:_2014}. For liver TG in random fed mice, ANCOVA model was used to estimate the effect of PPP2R5C knockdown by \gls{AAV}. Two variables liver \gls{nefa} and \gls{AAV} (miRNC and miR12) were used as covariates to predict the liver TG. In \gls{gtt} measurement, repeated measures ANOVA model was used to evaluated the effect of PPP2R5C knockdown on glucose clearance rate. For other metabolite measurements, pairwise.t.test function in R was employed to perform multiple comparisons between 3 different treatments (Random Fed, Fasting, Refed) and 2 different \gls{AAV}s (miRNC, miR12). p-value form multiple testing was controlled by Benjamini \& Hochberg method \cite{benjamini_controlling_1995}. All the plotting was done either using R core graphics or ggplot2  \cite{wickham_ggplot2:_2009}.






 


