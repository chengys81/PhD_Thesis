% !TEX encoding = UTF-8 Unicode

% This file contains the German version of your abstract, with about 300-500 words
Das Verständnis des Entstehungsmechanismus von Krebs und Stoffwechselkrankheiten ist von gro{\ss}er Wichtigkeit um neue Behandlungen für diese Krankheiten zu entwickeln. Unter den verschiedenen Signalwegen, welche in diesen Krankheiten involviert sind, spielt Insulinsignalisierung eine bedeutende Rolle in der Modulierung von Zellwachstum und der Stoffwechselrate. Die bisherigen Forschungsergebnisse unseres Labors \cite{hahn_pp2a_2010} zeigen dass die \textit{Drosophila} PP2A regulatorische Untereinheit B\textsuperscript{$\prime$} ein negativer Regulator von S6K ist; eine bedeutende downstream Komponente der Insulin und mTOR Regulierung. Um die Funktion dieser regulatorischen Untereinheit und ihrer Auswirkung auf Translationale Medizin besser zu verstehen, werden Mausmodelle und Zellkulturen zur Charakterisierung der Funktion von Säugetier Homolog PPP2R5C in Mäusen verwendet--welches sich als molekular Checkpoint zur Regulierung der Balance zwischen Glukose und Lipid Homöostase in Mausleber erwies. Knockdown von PPP2R5C in Hepa 1-6 und Primären Maus-Hepatozyten zeigt dass PPP2R5C ein negativer Regulator für Triglyceridespeicherung und Glykolyse sein könnte. Knockdown von PPP2R5C in mehreren Maus Zelllinien resultiert in erhöhter Glukoseaufnahme und Glykolyserate. Knockdown von PPP2R5C, insbesondere in Mausleber, verändert den Maus Metabolismus auf dramatische Weise. Leber Triglycerid und Glykogen werden gesteigert und Leber Cholesterin vermindert. Trotz keiner Veränderung im Blutglukosespiegel haben die Knockdown Mäuse besser Insulinsensitivität und Glukosetoleranz. Zwischen fasten und füttern haben die Knockdown Mäuse auch erhöhte VLDL Aussonderungen der Leber. Microarray und \gls{qpcr} Analyse zeigt auch dass mehrere Gene; die in Glykolyse und Lipogenese involviert sind, nach PPP2R5C Knockdown hochgeregelt werden. Den meisten dieser Gene könnte eine Steigerung von HIF1\textalpha{} und SREBP-1 Aktivität zugeschrieben werden. PPP2R5C Substrat-Trapping identifiziert mehrere Hauptregulatoren des Stoffwechselvorgangs, wie etwa AMPK, HIF1\textalpha{} und STAT3. \textit{In vitro} Knockdown von PPP2R5C zeigt auch erhöhte AMPK Aktivität und erhöhte HIF1\textalpha{} Phosphorylierung auf. Interessanterweise ist Maus PPP2R5C in \textit{db/db} Mausleber hochgeregelt, welches ein Mausmodell von Typ 2 Diabetes ist. Des Weiteren ist menschliche PPP2R5C in der Leber von Diabetes Patienten auch erhöht.
